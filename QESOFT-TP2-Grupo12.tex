% !TEX encoding = UTF-8 Unicode

%=============================================
%---(C) QESOFT, Isabel Sampaio (ais@isep.ipp.pt), 2019 ---
%=============================================

%----------tipo de documento------------------l
\documentclass[openany,10pt,a4paper]{article}

% ----------------------
% Configuração página

\usepackage
[
a4paper,
left=2.6cm,
right=2.4cm,
top= 2.6 cm,
bottom=1.5 cm,
]
{geometry}


% ---- Sem indentação parágrafo -------
\setlength\parindent{0pt} 
\setlength{\voffset}{-0.2in}

% ---- Espaçamento entre parágrafos -------
\setlength{\parskip}{5pt}

% -----Topo------------------
\setlength{\topmargin}{-\headheight}
\setlength{\headsep}{0.0cm}

% -----Espaçamento palavras---------------
\renewcommand{\baselinestretch}{1.0}   

%------ Tabelas -----
\usepackage{longtable}

% ------------------ Opções em tabelas-----------------
\usepackage{array}		% para centrar colunas (de tabelas) com p{XXcm}
\usepackage{paralist}	% para colocar enumerates em linha com um parágrafo

\usepackage{booktabs,multirow} 
\usepackage{longtable}
\usepackage{tabularx}
\usepackage{subcaption}

\usepackage{tabu}
\usepackage{tabulary}

% -----------------PORTUGUES----------
\usepackage[utf8]{inputenc}
\usepackage[T1]{fontenc}

%-----tipo letra---------
\usepackage{helvet} % --- não existe Calibri em  pdflatex
\renewcommand{\familydefault}{\sfdefault}

%---Portuguese-specific commands ----
\usepackage[english,portuguese]{babel}
 
% -----------------   Hyphenation rules ----------------
\usepackage{hyphenat}
\hyphenation{re-cu-pe-rar te-ma ma-té-ri-a ci-en-tí-fi-co}

%-------------------------------------------------------------------------------------------------------
%para palavras que ao quebrar leva 2 hifen ,: por exemplo "verificou-se": no texto fazer verificou{-}{-}se
\defineshorthand{"-}{\nobreak\hskip0pt\discretionary{-}{-}{-}\nobreak\hskip0pt} 

% ---------- Hyperlinks ------------
\usepackage{cooltooltips} 
\def\cool{\texttt{cool}}

\usepackage{verbatim}

% -------- Incluir pdfs-------
\usepackage{graphicx}

%---------Para Titulos: capitulos, secções, etc.-----------
\usepackage{titlesec}

\titleformat{\section} {\Large\bfseries}{\thesection.}{1em}{}
\titleformat*{\subsection}{\large}
\titleformat*{\subsubsection}{\large}

%------------- espaçamento antes/depois titulos -----------

\titlespacing{\section}{0pt}{*1.1}{*1.0}
\titlespacing{\subsection}{0pt}{*1.1}{*1.0}
\titlespacing{\subsubsection}{0pt}{*1.1}{*1.0}

 %--------- Inicio do documento--------------------
\begin{document}
\pagenumbering{arabic}

\title{\textbf{Título}}
\author{João Cardoso, \textit{1150943} \\ Sofia Silva, \textit{1150690} \\ Tiago Leite, \textit{1150780}}
\date{}

\maketitle

% -------------- SECÇÕES ---------------
\section{Introdução}

Para este relatório interessam apenas aspectos técnico/científicos. Indicar o objectivo. Existem unicamente 3 níveis para as secções. Não podem existir linhas em branco (parágrafos vazios). O relatório tem de seguir obrigatoriamente este modelo, logo não pode ser alterado, exceptuando-se pela adição de (sub)secções que os autores considerem ser necessárias, mas sempre sem ultrapassar os 3 níveis como referido. A introdução nunca poderá ultrapassar a primeira página.

Utilizar figuras e tabelas de forma criteriosa. O conteúdo das tabelas tem de utilizar a mesma fonte do texto. O relatório não pode possuir menos de 10 páginas nem mais de 12, excluindo referências. 
Poderão existir anexos, mesmo para lá do limite do documento, mas em princípio o seu conteúdo não deverá ser avaliado (excepção abaixo). A numeração dos anexos é alfabética e manual. 

Os autores devem ser identificados pelo primeiro e último nomes, e pelo número. O número deve encontrar-se em itálico.

\subsection{Secção Nível 2}
Nível a usar quando necessário.

\subsubsection{Secção Nível 3}
Nível a usar apenas quando for estritamente necessário.

\section{Escolha do Projecto}
O relatório pode seguir a estrutura do próprio trabalho. Deverá existir uma secção que apresente o projecto escolhido para ser avaliado e a justificação da escolha desse projecto.

\section{Avaliação e Outras Secções}
O relatório deve descrever toda a avaliação realizada e os resultados dessa avaliação. A avaliação do processo terá de ser realizada usando o CMMI \cite{CMMI2010} e de acordo com a abordagem indicada nas aulas e complementada com os documentos fornecidos, incluindo este. A avaliação do produto seguirá o modelo OMG a ser indicado. Usar uma secção para cada avaliação.

O trabalho requer o desenvolvimento de um questionário para cada avaliação. Os questionários devem ser desenvolvidos de forma rigorosa como qualquer questionário usado em investigação, conforme referido nas aulas e documentos fornecidos. Documentar devidamente o desenvolvimento dos questionários em secção específica para o efeito. Os questionários ficarão em anexo e/ou online.

Cada grupo tem de avaliar um projecto escolhido por si e que possua “fases” razoavelmente definidas, tal como indicado no enunciado.

Apresentar muito sucintamente as principais limitações acerca da forma como as avaliações foram realizadas. Incluir uma secção com a identificação das áreas do processo/produto a melhorar ordenadas por prioridade, e devidamente justificado. Nessa secção ou noutra deverá ser feito a análise integrada das duas avaliações. A aplicação das ferramentas deve ser resumida numa página, máximo duas páginas.

\section{Conclusões}

As principais conclusões serão apresentadas aqui. Notar que apenas interessam aspectos técnico/ científicos. Quaisquer dúvidas terão de ser colocadas ao docente.

% -------------- BIBLIOGRAFIA -------------
\renewcommand\bibname{Referências} % mudar nome capítulo 
\addcontentsline{toc}{chapter}{Referências}
\bibliographystyle{ieee} 
\bibliography{bibliografia/Referencias}   

% --------------- ANEXOS -----------------
\begin{appendix}

\section*{Anexo}\label{ConteudoAnexos}

Aqui são apresentados exemplos de figuras e tabelas.

Exemplo de uma figura:\\

\begin{figure}[h]
	\centering
	\includegraphics[width=0.4\linewidth]{figuras/CMMI.pdf}
	\caption{Exemplo de imagem: \small{\textit{Fonte: 	\cite{CMMI2010}}}}
	\label{fig:figura1}
\end{figure}


Exemplo de uma tabela:\\

\begin{table}[h]
	\centering
	\caption{Exemplo de tabela.}
	\begin{tabular}{p{1.5in}p{1.5in}}		
		\toprule
	
		\textbf{UC's}  & 	 \textbf{Ano} \\ 
		\midrule
		QESOFT & Primeiro \\
		Tese & Segundo \\
		\bottomrule
	\end{tabular} 
	\label{tab:tabela1}
\end{table}

\end{appendix}
\end{document}

