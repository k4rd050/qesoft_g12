% !TEX encoding = UTF-8 Unicode

%=============================================
%---(C) QESOFT, Isabel Sampaio (ais@isep.ipp.pt), 2019 ---
%=============================================

%----------tipo de documento------------------
\documentclass[openany,10pt,a4paper]{article}

% ----------------------
% Configuração página

\usepackage
[
a4paper,
left=2.6cm,
right=2.4cm,
top= 2.6 cm,
bottom=1.5 cm,
]
{geometry}


% ---- Sem indentação parágrafo -------
\setlength\parindent{0pt} 
\setlength{\voffset}{-0.2in}

% ---- Espaçamento entre parágrafos -------
\setlength{\parskip}{5pt}

% -----Topo------------------
\setlength{\topmargin}{-\headheight}
\setlength{\headsep}{0.0cm}

% -----Espaçamento palavras---------------
\renewcommand{\baselinestretch}{1.0}   

%------ Tabelas -----
\usepackage{longtable}

% ------------------ Opções em tabelas-----------------
\usepackage{array}		% para centrar colunas (de tabelas) com p{XXcm}
\usepackage{paralist}	% para colocar enumerates em linha com um parágrafo

\usepackage{booktabs,multirow} 
\usepackage{longtable}
\usepackage{tabularx}
\usepackage{subcaption}

\usepackage{tabu}
\usepackage{tabulary}

% -----------------PORTUGUES----------
\usepackage[utf8]{inputenc}
\usepackage[T1]{fontenc}

%-----tipo letra---------
\usepackage{helvet} % --- não existe Calibri em  pdflatex
\renewcommand{\familydefault}{\sfdefault}

%---Portuguese-specific commands ----
\usepackage[english,portuguese]{babel}
 
% -----------------   Hyphenation rules ----------------
\usepackage{hyphenat}
\hyphenation{re-cu-pe-rar te-ma ma-té-ri-a ci-en-tí-fi-co}

%-------------------------------------------------------------------------------------------------------
%para palavras que ao quebrar leva 2 hifen ,: por exemplo "verificou-se": no texto fazer verificou{-}{-}se
\defineshorthand{"-}{\nobreak\hskip0pt\discretionary{-}{-}{-}\nobreak\hskip0pt} 

% ---------- Hyperlinks ------------
\usepackage{cooltooltips} 
\def\cool{\texttt{cool}}

\usepackage{verbatim}

% -------- Incluir pdfs-------
\usepackage{graphicx}

%---------Para Titulos: capitulos, secções, etc.-----------
\usepackage{titlesec}

\titleformat{\section} {\Large\bfseries}{\thesection.}{1em}{}
\titleformat*{\subsection}{\large}
\titleformat*{\subsubsection}{\large}

%------------- espaçamento antes/depois titulos -----------

\titlespacing{\section}{0pt}{*1.1}{*1.0}
\titlespacing{\subsection}{0pt}{*1.1}{*1.0}
\titlespacing{\subsubsection}{0pt}{*1.1}{*1.0}

 %--------- Inicio do documento--------------------
\begin{document}
\pagenumbering{arabic}

\title{\textbf{Uma visão sobre a Qualidade de Processo e de Produto em ambiente académico}}
\author{João Cardoso, \textit{1150943} \\ Sofia Silva, \textit{1150690} \\ Tiago Leite, \textit{1150780}}
\date{}

\maketitle

% -------------- SECÇÕES ---------------
\section{Introdução}

Para este relatório interessam apenas aspectos técnico/científicos. Indicar o objectivo. Existem unicamente 3 níveis para as secções. Não podem existir linhas em branco (parágrafos vazios). O relatório tem de seguir obrigatoriamente este modelo, logo não pode ser alterado, exceptuando-se pela adição de (sub)secções que os autores considerem ser necessárias, mas sempre sem ultrapassar os 3 níveis como referido. A introdução nunca poderá ultrapassar a primeira página.

Utilizar figuras e tabelas de forma criteriosa. O conteúdo das tabelas tem de utilizar a mesma fonte do texto. O relatório não pode possuir menos de 10 páginas nem mais de 12, excluindo referências. 
Poderão existir anexos, mesmo para lá do limite do documento, mas em princípio o seu conteúdo não deverá ser avaliado (excepção abaixo). A numeração dos anexos é alfabética e manual. 

Os autores devem ser identificados pelo primeiro e último nomes, e pelo número. O número deve encontrar-se em itálico.

\subsection{Secção Nível 2}
Nível a usar quando necessário.

\subsubsection{Secção Nível 3}
Nível a usar apenas quando for estritamente necessário.

\section{Escolha do Projecto}
O relatório pode seguir a estrutura do próprio trabalho. Deverá existir uma secção que apresente o projecto escolhido para ser avaliado e a justificação da escolha desse projecto.

\section{Avaliação e Outras Secções}
O relatório deve descrever toda a avaliação realizada e os resultados dessa avaliação. A avaliação do processo terá de ser realizada usando o CMMI e de acordo com a abordagem indicada nas aulas e complementada com os documentos fornecidos, incluindo este. A avaliação do produto seguirá o modelo OMG a ser indicado. Usar uma secção para cada avaliação.

O trabalho requer o desenvolvimento de um questionário para cada avaliação. Os questionários devem ser desenvolvidos de forma rigorosa como qualquer questionário usado em investigação, conforme referido nas aulas e documentos fornecidos. Documentar devidamente o desenvolvimento dos questionários em secção específica para o efeito. Os questionários ficarão em anexo e/ou online.

Cada grupo tem de avaliar um projecto escolhido por si e que possua “fases” razoavelmente definidas, tal como indicado no enunciado.

Apresentar muito sucintamente as principais limitações acerca da forma como as avaliações foram realizadas. Incluir uma secção com a identificação das áreas do processo/produto a melhorar ordenadas por prioridade, e devidamente justificado. Nessa secção ou noutra deverá ser feito a análise integrada das duas avaliações. A aplicação das ferramentas deve ser resumida numa página, máximo duas páginas.

\subsection{Qualidade do Processo}
Nesta subsecção é apresentada, tal como pedido, uma análise informal ao processo utilizado no decorrer do projeto escolhido (LAPR4). Enquadra-se primeiramente o conceito de qualidade de processo e um exemplo da sua aplicação num caso real com recurso ao CMMI For Development.
Os modelos CMMI(\textit{Capability Maturity Model Integration}), desenvolvidos por equipas multidisciplinares da Universidade de Carnegie Mellon, incluindo o SEI (\textit{Software Engineering Institute}), são coleções de boas práticas que procuram ajudar as organizações a melhorar os seus processos. O modelo CMMI-DEV (CMMI for Development) é especialmente ligado aos processos de desenvolvimento de produtos e serviços, daí a sua utilização no meio do desenvolvolvimento de software. Este possui orientações com foco na melhoria da qualidade final do produto para cumprir as necessidades dos consumidores finais. \cite{CMMIProductTeam2010} 
A análise da qualidade de processo a seguir apresentada teve como base uma representação discreta com foco nos níveis 2 e 3 de maturidade e em 7 e 11 áreas de processo para cada um desses níveis respetivamente. Para a obtenção de resultados foi elaborado um questionário (em anexo) com um conjunto questões para cada uma das áreas de processo, estando dessa forma dividido. Para cada um dos processos existem 4 possíveis respostas: Totalmente Implementado, Implementado Parcialmente, Não Implementado ou Não Sei caso não seja aplicável ou seja desconhecido.

\subsubsection{Resultados}
Nesta subsecção são apresentados os resultados relativos ao questionário desenvolvido. \\
De forma a facilitar a análise dos resultados, foi organizado o resumo para cada área de processo, apresentando o seu objetivo, as metas definidas, assim como as respetivas avaliações seguindo os critérios definidos (também em anexo). \\

\textbf{Gestão de Configuração} \\
\textbf{Objetivo:} Permitir a integração contínua dos vários componentes de software em que as mudanças são devidamente documentadas \\

\begin{table}[]
    \centering
    \caption{Gestão de Configuração - Resultados}
    \begin{tabular}{p{2in}|p{2in}|p{2in}}
        \hline	
		\textbf{Meta} & \textbf{Questões} & \textbf{Avaliação} \\ \hline
		Identificar e definir as configurações disponíveis & 1 e 2 & TODO \\ \hline
		Monitorizar as alterações de configuração  & 3 e 4 & TODO \\ \hline
		Documentar as alterações de configuração  & 5, 6 e 7 & TODO \\ \hline
    \end{tabular}
    \label{tab:analise-gestconfig}
\end{table}

\textbf{Medição e Análise}  \\
\textbf{Objetivo:} Permitir à organização realizar uma medição da sua qualidade

\begin{table}[]
    \centering
    \caption{Medição e Análise - Resultados}
    \begin{tabular}{p{2in}|p{2in}|p{2in}}
         \hline	
		\textbf{Meta} & \textbf{Questões} & \textbf{Avaliação} \\ \hline
		Definir mecanismos de medição & 8, 9, 10, 11 & TODO \\ \hline
		Pôr em prática os mecanismos definidos  & 12, 13 & TODO \\ \hline
		Os mecanismos são utilizados ativamente na organização & 14 e 15 & TODO \\ \hline
    \end{tabular}
    \label{tab:analise-medicaoanalise}
\end{table}

\textbf{Monitorização e Controlo do Projeto}  \\
\textbf{Objetivo:} Permitir à organização uma monitorização do progresso dos projetos em curso
    
\begin{table}[]
    \centering
    \caption{Monitorização e Controlo do Projeto - Resultados}
    \begin{tabular}{p{2in}|p{2in}|p{2in}}
         \hline	
		\textbf{Meta} & \textbf{Questões} & \textbf{Avaliação} \\ \hline
		Criar métricas de monitorização de progresso de projetos em relação ao planeado & 17, 18, 19 e 20 & TODO \\ \hline
		Utilizar o controlo do progresso dos projetos para organizar retrospetivas & 21, 22, 23 & TODO \\ \hline
		Aplicar efetivamente medidas corretivas para os processos negativamente identificados na monitorização & 24 e 25 & TODO \\ \hline
    \end{tabular}
    \label{tab:analise-monitorizacao}
\end{table}

\textbf{Planeamento de Projeto}  \\
\textbf{Objetivo:} Melhorar a fase de planeamento de projetos da organização

\begin{table}[]
    \centering
    \caption{Planeamento de Projeto - Resultados}
    \begin{tabular}{p{3in}|p{2in}|p{1in}}
        \hline	
		\textbf{Meta} & \textbf{Questões} & \textbf{Avaliação} \\ \hline
		Definir a estrutura de estimativa do planeamento & 26 e 27 & TODO \\ \hline
		As estimativas são eficazes e mantidas ao longo do projeto & 29 e 30 & TODO \\ \hline
		Toda a estrutura definida para o planeamento é identificada, analisada e monitorizada & 31, 32, 33, 34, 35 e 36 & TODO \\ \hline
		Há uma revisão e apoio à execução do planeamento efetuado & 37, 38 e 39
	\end{tabular}
    \label{tab:analise-planeamento}
\end{table}



\subsection{Qualidade do Produto}
De modo a avaliar a qualidade do produto, foi definido o questionário presente no anexo \textbf{X}, que se baseou na especificação dos documentos do Object Management Group (OMG). Assim, foram selecionadas 5 classes aleatórias de modo a contabilizar o número de ocorrencia de cada regra definida no questionário. 



\begin{table}[h]
	\centering
	\caption{Regras da caraterística manutenibilidade}
	\begin{tabular}{|p{3in}|p{0.3in}|p{0.3in}|p{0.3in}|p{0.3in}|p{0.3in}|p{0.4in}|}
		\hline	
		\textbf{Regra} & \textbf{C1} & \textbf{C2} & \textbf{C3} & \textbf{C4} & \textbf{C5} & \textbf{Média} \\ \hline
		\textbf{ASCMM-MNT-1}: Control Flow Transfer Control Element outside Switch Block & 0 & 0 & 0 & 0 & 0 & 0 \\ \hline
\textbf{ASCMM-MNT-2}: Class Element Excessive Inheritance of Class Elements with Concrete Implementation & 0 & 0 & 0 & 0 & 0 & 0 \\ \hline
\textbf{ASCMM-MNT-3}: Storable and Member Data Element Initialization with Hard-Coded Literals & 0 & 0 & 0 & 0 & 0 & 0 \\ \hline
\textbf{ASCMM-MNT-4}: Callable and Method Control Element Number of Outward Calls & 0 & 0 & 0 & 0 & 0 & 0 \\ \hline
\textbf{ASCMM-MNT-5}: Loop Value Update within the Loop & 0 & 0 & 0 & 0 & 0 & 0 \\ \hline
\textbf{ASCMM-MNT-6}: Commented-out Code Element Excessive Volume & 0 & 0 & 0 & 0 & 0 & 0 \\ \hline
\textbf{ASCMM-MNT-7}: Inter-Module Dependency Cycles & 0 & 0 & 0 & 0 & 0 & 0 \\ \hline
\textbf{ASCMM-MNT-8}: Source Element Excessive Size & 0 & 0 & 0 & 0 & 0 & 0 \\ \hline
\textbf{ASCMM-MNT-9}: Horizontal Layer Excessive Number & 0 & 0 & 0 & 0 & 0 & 0 \\ \hline
\textbf{ASCMM-MNT-10}: Named Callable and Method Control Element Multi-Layer Span & 0 & 0 & 0 & 0 & 0 & 0 \\ \hline
\textbf{ASCMM-MNT-11}: Callable and Method Control Element Excessive Cyclomatic Complexity Value & 0 & 0 & 0 & 0 & 0 & 0 \\ \hline
\textbf{ASCMM-MNT-12}: Named Callable and Method Control Element with Layer-skipping Call & 0 & 0 & 0 & 0 & 0 & 0 \\ \hline
\textbf{ASCMM-MNT-13}: Callable and Method Control Element Excessive Number of Parameters & 0 & 0 & 0 & 0 & 0 & 0 \\ \hline
\textbf{ASCMM-MNT-14}: Callable and Method Control Element Excessive Number of Control Elements involving Data Element from Data Manager or File Resource & 0 & 0 & 0 & 0 & 0 & 0 \\ \hline
\textbf{ASCMM-MNT-15}: Public Member Element & 0 & 0 & 0 & 0 & 0 & 0 \\ \hline
\textbf{ASCMM-MNT-16}: Method Control Element Usage of Member Element from other Class Element & 0 & 0 & 0 & 0 & 0 & 0 \\ \hline
\textbf{ASCMM-MNT-17}: Class Element Excessive Inheritance Level & 0 & 0 & 0 & 0 & 0 & 0 \\ \hline
\textbf{ASCMM-MNT-18}: Class Element Excessive Number of Children & 0 & 0 & 0 & 0 & 0 & 0 \\ \hline
\textbf{ASCMM-MNT-19}: Named Callable and Method Control Element Excessive Similarity & 0 & 0 & 0 & 0 & 0 & 0 \\ \hline
\textbf{ASCMM-MNT-20}: Unreachable Named Callable or Method Control Element & 0 & 0 & 0 & 0 & 0 & 0 \\ \hline
	\end{tabular} 
	\label{tab:tabela1}
\end{table}


\begin{table}[h]
	\centering
	\caption{Regras da caraterística segurança}
	\begin{tabular}{|p{3in}|p{0.3in}|p{0.3in}|p{0.3in}|p{0.3in}|p{0.3in}|p{0.4in}|}
		\hline	
		\textbf{Regra} & \textbf{C1} & \textbf{C2} & \textbf{C3} & \textbf{C4} & \textbf{C5} & \textbf{Média} \\ \hline
		\textbf{ASCMM-CWE-22}: Path Traversal Improper Input Neutralization & 0 & 0 & 0 & 0 & 0 & 0 \\ \hline
		\textbf{ASCMM-CWE-78}: OS Command Injection Improper Input Neutralization & 0 & 0 & 0 & 0 & 0 & 0 \\ \hline
		\textbf{ASCMM-CWE-79}: Cross-site Scripting Improper Input Neutralization & 0 & 0 & 0 & 0 & 0 & 0 \\ \hline
		\textbf{ASCMM-CWE-89}: SQL Injection Improper Input Neutralization & 0 & 0 & 0 & 0 & 0 & 0 \\ \hline
		\textbf{ASCMM-CWE-99}: Name or Reference Resolution Improper Input Neutralization & 0 & 0 & 0 & 0 & 0 & 0 \\ \hline
		\textbf{ASCMM-CWE-120}: Buffer Copy without Checking Size of Input & 0 & 0 & 0 & 0 & 0 & 0 \\ \hline
		\textbf{ASCMM-CWE-129}: Array Index Improper Input Neutralization & 0 & 0 & 0 & 0 & 0 & 0 \\ \hline
		\textbf{ASCMM-CWE-134}: Format String Improper Input Neutralization & 0 & 0 & 0 & 0 & 0 & 0 \\ \hline
		\textbf{ASCMM-CWE-252-resource}: Unchecked Return Parameter Value of named Callable and Method Control Element with Read, Write, and Manage Access to Platform Resource & 0 & 0 & 0 & 0 & 0 & 0 \\ \hline
		\textbf{ASCMM-CWE-327}: Broken or Risky Cryptographic Algorithm Usage & 0 & 0 & 0 & 0 & 0 & 0 \\ \hline
		\textbf{ASCMM-CWE-396}: Declaration of Catch for Generic Exception & 0 & 0 & 0 & 0 & 0 & 0 \\ \hline
		\textbf{ASCMM-CWE-397}: Declaration of Throws for Generic Exception & 0 & 0 & 0 & 0 & 0 & 0 \\ \hline
		\textbf{ASCMM-CWE-434}: File Upload Improper Input Neutralization & 0 & 0 & 0 & 0 & 0 & 0 \\ \hline
		\textbf{ASCMM-CWE-456}: Storable and Member Data Element Missing Initialization & 0 & 0 & 0 & 0 & 0 & 0 \\ \hline
		\textbf{ASCMM-CWE-606}: Unchecked Input for Loop Condition & 0 & 0 & 0 & 0 & 0 & 0 \\ \hline
		\textbf{ASCMM-CWE-667}: Shared Resource Improper Locking & 0 & 0 & 0 & 0 & 0 & 0 \\ \hline
		\textbf{ASCMM-CWE-672}: Expired or Released Resource Usage & 0 & 0 & 0 & 0 & 0 & 0 \\ \hline
		\textbf{ASCMM-CWE-681}: Numeric Types Incorrect Conversion & 0 & 0 & 0 & 0 & 0 & 0 \\ \hline
		\textbf{ASCMM-CWE-772}: Missing Release of Resource after Effective Lifetime & 0 & 0 & 0 & 0 & 0 & 0 \\ \hline
		\textbf{ASCMM-CWE-789}: Uncontrolled Memory Allocation & 0 & 0 & 0 & 0 & 0 & 0 \\ \hline
		\textbf{ASCMM-CWE-798}: Hard-Coded Credentials Usage for Remote Authentication & 0 & 0 & 0 & 0 & 0 & 0 \\ \hline
		\textbf{ASCMM-CWE-835}: Loop with Unreachable Exit Condition ('Infinite Loop') & 0 & 0 & 0 & 0 & 0 & 0 \\ \hline
	\end{tabular} 
	\label{tab:tabela1}
\end{table}


\begin{table}[h]
	\centering
	\caption{Regras da caraterística eficiencia de performance}
	\begin{tabular}{|p{3in}|p{0.3in}|p{0.3in}|p{0.3in}|p{0.3in}|p{0.3in}|p{0.4in}|}
		\hline	
		\textbf{Regra} & \textbf{C1} & \textbf{C2} & \textbf{C3} & \textbf{C4} & \textbf{C5} & \textbf{Média} \\ \hline
		\textbf{ASCPEM-PRF-1}: Static Block Element containing Class Instance Creation Control Element & 0 & 0 & 0 & 0 & 0 & 0 \\ \hline
\textbf{ASCPEM-PRF-2}: Immutable Storable and Member Data Element Creation & 0 & 0 & 0 & 0 & 0 & 0 \\ \hline
\textbf{ASCPEM-PRF-3}: Static Member Data Element outside of a Singleton Class Element & 0 & 0 & 0 & 0 & 0 & 0 \\ \hline
\textbf{ASCPEM-PRF-4}: Data Resource Read and Write Access Excessive Complexity & 0 & 0 & 0 & 0 & 0 & 0 \\ \hline
\textbf{ASCPEM-PRF-5}: Data Resource Read Access Unsupported by Index Element & 0 & 0 & 0 & 0 & 0 & 0 \\ \hline
\textbf{ASCPEM-PRF-6}: Large Data Resource ColumnSet Excessive Number of Index Elements & 0 & 0 & 0 & 0 & 0 & 0 \\ \hline
\textbf{ASCPEM-PRF-7}: Large Data Resource ColumnSet with Index Element of Excessive Size & 0 & 0 & 0 & 0 & 0 & 0 \\ \hline
\textbf{ASCPEM-PRF-8}: Control Elements Requiring Significant Resource Element within Control Flow Loop Block & 0 & 0 & 0 & 0 & 0 & 0 \\ \hline
\textbf{ASCPEM-PRF-9}: Non-stored SQL Callable Control Element with Excessive Number of Data Resource Access & 0 & 0 & 0 & 0 & 0 & 0 \\ \hline
\textbf{ASCPEM-PRF-10}: Non-SQL Named Callable and Method Control Element with Excessive Number of Data Resource Access & 0 & 0 & 0 & 0 & 0 & 0 \\ \hline
\textbf{ASCPEM-PRF-11}: Data Access Control Element from Outside Designated Data Manager Component & 0 & 0 & 0 & 0 & 0 & 0 \\ \hline
\textbf{ASCPEM-PRF-12}: Storable and Member Data Element Excessive Number of Aggregated Storable and Member Data Elements & 0 & 0 & 0 & 0 & 0 & 0 \\ \hline
\textbf{ASCPEM-PRF-13}: Data Resource Access not using Connection Pooling Capability & 0 & 0 & 0 & 0 & 0 & 0 \\ \hline
\textbf{ASCPEM-PRF-14}: Storable and Member Data Element Memory Allocation Missing De-allocation Control Element & 0 & 0 & 0 & 0 & 0 & 0 \\ \hline
\textbf{ASCPEM-PRF-15}: Storable and Member Data Element Reference Missing De-referencing Control Element & 0 & 0 & 0 & 0 & 0 & 0 \\ \hline
	\end{tabular} 
	\label{tab:tabela1}
\end{table}

\begin{table}[h]
	\centering
	\caption{Regras da caraterística eficiencia de performance}
	\begin{tabular}{|p{3in}|p{0.3in}|p{0.3in}|p{0.3in}|p{0.3in}|p{0.3in}|p{0.4in}|}
		\hline	
		\textbf{Regra} & \textbf{C1} & \textbf{C2} & \textbf{C3} & \textbf{C4} & \textbf{C5} & \textbf{Média} \\ \hline
		\textbf{ASCRM-CWE-252-data}: Unchecked Return Parameter Value of named Callable and Method Control Element with Read, Write, and Manage Access to Data Resource & 0 & 0 & 0 & 0 & 0 & 0 \\ \hline
\textbf{ASCRM-CWE-252-resource}: Unchecked Return Parameter Value of named Callable and Method Control Element with Read, Write, and Manage Access to Platform Resource & 0 & 0 & 0 & 0 & 0 & 0 \\ \hline
\textbf{ASCRM-CWE-396}: Declaration of Catch for Generic Exception & 0 & 0 & 0 & 0 & 0 & 0 \\ \hline
\textbf{ASCRM-CWE-397}: Declaration of Throws for Generic Exception & 0 & 0 & 0 & 0 & 0 & 0 \\ \hline
\textbf{ASCRM-CWE-456}: Storable and Member Data Element Missing Initialization & 0 & 0 & 0 & 0 & 0 & 0 \\ \hline
\textbf{ASCRM-CWE-674}: Uncontrolled Recursion & 0 & 0 & 0 & 0 & 0 & 0 \\ \hline
\textbf{ASCRM-CWE-704}: Incorrect Type Conversion or Cast & 0 & 0 & 0 & 0 & 0 & 0 \\ \hline
\textbf{ASCRM-CWE-772}: Missing Release of Resource after Effective Lifetime & 0 & 0 & 0 & 0 & 0 & 0 \\ \hline
\textbf{ASCRM-CWE-788}: Memory Location Access After End of Buffer & 0 & 0 & 0 & 0 & 0 & 0 \\ \hline
\textbf{ASCRM-RLB-1}: Empty Exception Block & 0 & 0 & 0 & 0 & 0 & 0 \\ \hline
\textbf{ASCRM-RLB-2}: Serializable Storable Data Element without Serialization Control Element & 0 & 0 & 0 & 0 & 0 & 0 \\ \hline
\textbf{ASCRM-RLB-3}: Serializable Storable Data Element with non-Serializable Item Elements & 0 & 0 & 0 & 0 & 0 & 0 \\ \hline
\textbf{ASCRM-RLB-4}: Persistent Storable Data Element without Proper Comparison Control Element & 0 & 0 & 0 & 0 & 0 & 0 \\ \hline
\textbf{ASCRM-RLB-5}: Runtime Resource Management Control Element in a Component Built to Run on Application Servers & 0 & 0 & 0 & 0 & 0 & 0 \\ \hline
\textbf{ASCRM-RLB-6}: Storable or Member Data Element containing Pointer Item Element without Proper Copy Control Element & 0 & 0 & 0 & 0 & 0 & 0 \\ \hline
\textbf{ASCRM-RLB-7}: Class Instance Self Destruction Control Element & 0 & 0 & 0 & 0 & 0 & 0 \\ \hline
\textbf{ASCRM-RLB-8}: Named Callable and Method Control Elements with Variadic Parameter Element & 0 & 0 & 0 & 0 & 0 & 0 \\ \hline
\textbf{ASCRM-RLB-9}: Float Type Storable and Member Data Element Comparison with Equality Operator & 0 & 0 & 0 & 0 & 0 & 0 \\ \hline
\textbf{ASCRM-RLB-10}: Data Access Control Element from Outside Designated Data Manager Component & 0 & 0 & 0 & 0 & 0 & 0 \\ \hline
\textbf{ASCRM-RLB-11}: Named Callable and Method Control Element in Multi-Thread Context with non-Final Static Storable or Member Element & 0 & 0 & 0 & 0 & 0 & 0 \\ \hline
\textbf{ASCRM-RLB-12}: Singleton Class Instance Creation without Proper Lock Element Management & 0 & 0 & 0 & 0 & 0 & 0 \\ \hline
\textbf{ASCRM-RLB-13}: Inter-Module Dependency Cycles & 0 & 0 & 0 & 0 & 0 & 0 \\ \hline
\textbf{ASCRM-RLB-14}: Parent Class Element with References to Child Class Element & 0 & 0 & 0 & 0 & 0 & 0 \\ \hline
\textbf{ASCRM-RLB-15}: Class Element with Virtual Method Element without Virtual Destructor & 0 & 0 & 0 & 0 & 0 & 0 \\ \hline
\textbf{ASCRM-RLB-16}: Parent Class Element without Virtual Destructor Method Element & 0 & 0 & 0 & 0 & 0 & 0 \\ \hline
\textbf{ASCRM-RLB-17}: Child Class Element without Virtual Destructor unlike its Parent Class Element & 0 & 0 & 0 & 0 & 0 & 0 \\ \hline
\textbf{ASCRM-RLB-18}: Storable and Member Data Element Initialization with Hard-Coded Network Resource Configuration Data & 0 & 0 & 0 & 0 & 0 & 0 \\ \hline
\textbf{ASCRM-RLB-19}: Synchronous Call Time-Out Absence & 0 & 0 & 0 & 0 & 0 & 0 \\ \hline
	\end{tabular} 
	\label{tab:tabela1}
\end{table}

\section{Conclusões}


As principais conclusões serão apresentadas aqui. Notar que apenas interessam aspectos técnico/ científicos. Quaisquer dúvidas terão de ser colocadas ao docente.

% -------------- BIBLIOGRAFIA -------------
\renewcommand\bibname{Referências} % mudar nome capítulo 
\addcontentsline{toc}{chapter}{Referências}
\setlength{\parskip}{0.7em}
\bibliographystyle{IEEEtran}
\bibliography{biblist.bib}

% --------------- ANEXOS -----------------
\begin{appendix}

\section*{Anexos}\label{ConteudoAnexos}

\subsection{Anexo 2}
\textbf{Questionário Qualidade do Produto}\\
De modo a analisar a avaliar a qualidade do produto, em foi desenvolvido o seguinte questionário que se encontra dividido em quatro grupos, representando medidas padrão: Manutenibilidade, Segurança, Eficiência da Performance e Fiabilidade.
Cada grupo contém um conjunto de regras de qualidade, para o qual terá de ser indicado o número de violações da mesma.

Grupo 1: Manutenibilidade
O propósito destas regras é estabelecer uma medida padrão de manutenibilidade com base na deteção de violações de boas práticas arquiteturais e de codificação que podem resultar em operações não confiáveis, como interrupções, corrupção de dados e longa recuperação de falhas do sistema.
\begin{itemize}
	\setlength\itemsep{0em}
	\item ASCMM-MNT-1: Control Flow Transfer Control Element outside Switch Block
	\item ASCMM-MNT-2: Class Element Excessive Inheritance of Class Elements with Concrete Implementation
	\item ASCMM-MNT-3: Storable and Member Data Element Initialization with Hard-Coded Literals
	\item ASCMM-MNT-4: Callable and Method Control Element Number of Outward Calls
	\item ASCMM-MNT-5: Loop Value Update within the Loop
	\item ASCMM-MNT-6: Commented-out Code Element Excessive Volume
	\item ASCMM-MNT-7: Inter-Module Dependency Cycles
	\item ASCMM-MNT-8: Source Element Excessive Size
	\item ASCMM-MNT-9: Horizontal Layer Excessive Number
	\item ASCMM-MNT-10: Named Callable and Method Control Element Multi-Layer Span
	\item ASCMM-MNT-11: Callable and Method Control Element Excessive Cyclomatic Complexity Value
	\item ASCMM-MNT-12: Named Callable and Method Control Element with Layer-skipping Call
	\item ASCMM-MNT-13: Callable and Method Control Element Excessive Number of Parameters
	\item ASCMM-MNT-14: Callable and Method Control Element Excessive Number of Control Elements involving Data Element from Data Manager or File Resource
	\item ASCMM-MNT-15: Public Member Element	
	\item ASCMM-MNT-16: Method Control Element Usage of Member Element from other Class Element
	\item ASCMM-MNT-17: Class Element Excessive Inheritance Level
	\item ASCMM-MNT-18: Class Element Excessive Number of Children
	\item ASCMM-MNT-19: Named Callable and Method Control Element Excessive Similarity
	\item ASCMM-MNT-20: Unreachable Named Callable or Method Control Element
\end{itemize}

Grupo 2: Segurança
O propósito destas regras é estabelecer uma medida padrão de segurança com base na deteção de violações de boas práticas arquiteturais e de codificação que poderão resultar na entrada não autorizada em sistemas, roubo de informações confidenciais, e o comprometimento malicioso da integridade do sistema.
\begin{itemize}
	\setlength\itemsep{0em}
	\item ASCSM-CWE-22: Path Traversal Improper Input Neutralization
	\item ASCSM-CWE-78: OS Command Injection Improper Input Neutralization
	\item ASCSM-CWE-79: Cross-site Scripting Improper Input Neutralization
	\item ASCSM-CWE-89: SQL Injection Improper Input Neutralization
	\item ASCSM-CWE-99: Name or Reference Resolution Improper Input Neutralization
	\item ASCSM-CWE-120: Buffer Copy without Checking Size of Input
	\item ASCSM-CWE-129: Array Index Improper Input Neutralization
	\item ASCSM-CWE-134: Format String Improper Input Neutralization
	\item ASCSM-CWE-252-resource: Unchecked Return Parameter Value of named Callable and Method Control Element with Read, Write, and Manage Access to Platform Resource
	\item ASCSM-CWE-327: Broken or Risky Cryptographic Algorithm Usage
	\item ASCSM-CWE-396: Declaration of Catch for Generic Exception
	\item ASCSM-CWE-397: Declaration of Throws for Generic Exception
	\item ASCSM-CWE-434: File Upload Improper Input Neutralization
	\item ASCSM-CWE-456: Storable and Member Data Element Missing Initialization
	\item ASCSM-CWE-606: Unchecked Input for Loop Condition
	\item ASCSM-CWE-667: Shared Resource Improper Locking
	\item ASCSM-CWE-672: Expired or Released Resource Usage
	\item ASCSM-CWE-681: Numeric Types Incorrect Conversion
	\item ASCSM-CWE-772: Missing Release of Resource after Effective Lifetime
	\item ASCSM-CWE-789: Uncontrolled Memory Allocation
	\item ASCSM-CWE-798: Hard-Coded Credentials Usage for Remote Authentication
	\item ASCSM-CWE-835: Loop with Unreachable Exit Condition ('Infinite Loop')
\end{itemize} 

Grupo 3: Eficiência de Performance
O propósito destas regras é estabelecer uma medida padrão de segurança com base na deteção de violações de boas práticas arquiteturais e de codificação que poderão resultar numa operação ineficiente, como a degradação do desempenho ou uso excessivo de recursos do processador.
\begin{itemize}	
\setlength\itemsep{0em}
	\item ASCPEM-PRF-1: Static Block Element containing Class Instance Creation Control Element
	\item ASCPEM-PRF-2: Immutable Storable and Member Data Element Creation
	\item ASCPEM-PRF-3: Static Member Data Element outside of a Singleton Class Element
	\item ASCPEM-PRF-4: Data Resource Read and Write Access Excessive Complexity
	\item ASCPEM-PRF-5: Data Resource Read Access Unsupported by Index Element
	\item ASCPEM-PRF-6: Large Data Resource ColumnSet Excessive Number of Index Elements
	\item ASCPEM-PRF-7: Large Data Resource ColumnSet with Index Element of Excessive Size
	\item ASCPEM-PRF-8: Control Elements Requiring Significant Resource Element within Control Flow Loop Block
	\item ASCPEM-PRF-9: Non-stored SQL Callable Control Element with Excessive Number of Data Resource Access
	\item ASCPEM-PRF-10: Non-SQL Named Callable and Method Control Element with Excessive Number of Data Resource Access
	\item ASCPEM-PRF-11: Data Access Control Element from Outside Designated Data Manager Component
	\item ASCPEM-PRF-12: Storable and Member Data Element Excessive Number of Aggregated Storable and Member Data Elements
	\item ASCPEM-PRF-13: Data Resource Access not using Connection Pooling Capability
	\item ASCPEM-PRF-14: Storable and Member Data Element Memory Allocation Missing De-allocation Control Element
	\item ASCPEM-PRF-15: Storable and Member Data Element Reference Missing De-referencing Control Element
\end{itemize}

Grupo 4: Fiabilidade
O propósito destas regras é estabelecer uma medida padrão de segurança com base na deteção de violações de boas práticas arquiteturais e de codificação que poderão resultar em operações não confiáveis, como interrupções, corrupção de dados e recuperação de falhas do sistema.
\begin{itemize}
	\setlength\itemsep{0em}
	\item ASCRM-CWE-252-data: Unchecked Return Parameter Value of named Callable and Method Control Element with Read, Write, and Manage Access to Data Resource
	\item ASCRM-CWE-252-resource: Unchecked Return Parameter Value of named Callable and Method Control Element with Read, Write, and Manage Access to Platform Resource
	\item ASCRM-CWE-396: Declaration of Catch for Generic Exception
	\item ASCRM-CWE-397: Declaration of Throws for Generic Exception
	\item ASCRM-CWE-456: Storable and Member Data Element Missing Initialization
	\item ASCRM-CWE-674: Uncontrolled Recursion
	\item ASCRM-CWE-704: Incorrect Type Conversion or Cast
	\item ASCRM-CWE-772: Missing Release of Resource after Effective Lifetime
	\item ASCRM-CWE-788: Memory Location Access After End of Buffer
	\item ASCRM-RLB-1: Empty Exception Block
	\item ASCRM-RLB-2: Serializable Storable Data Element without Serialization Control Element
	\item ASCRM-RLB-3: Serializable Storable Data Element with non-Serializable Item Elements
	\item ASCRM-RLB-4: Persistent Storable Data Element without Proper Comparison Control Element
	\item ASCRM-RLB-5: Runtime Resource Management Control Element in a Component Built to Run on Application Servers
	\item ASCRM-RLB-6: Storable or Member Data Element containing Pointer Item Element without Proper Copy Control Element
	\item ASCRM-RLB-7: Class Instance Self Destruction Control Element
	\item ASCRM-RLB-8: Named Callable and Method Control Elements with Variadic Parameter Element
	\item ASCRM-RLB-9: Float Type Storable and Member Data Element Comparison with Equality Operator
	\item ASCRM-RLB-10: Data Access Control Element from Outside Designated Data Manager Component
	\item ASCRM-RLB-11: Named Callable and Method Control Element in Multi-Thread Context with non-Final Static Storable or Member Element
	\item ASCRM-RLB-12: Singleton Class Instance Creation without Proper Lock Element Management
	\item ASCRM-RLB-13: Inter-Module Dependency Cycles
	\item ASCRM-RLB-14: Parent Class Element with References to Child Class Element
	\item ASCRM-RLB-15: Class Element with Virtual Method Element without Virtual Destructor
	\item ASCRM-RLB-16: Parent Class Element without Virtual Destructor Method Element
	\item ASCRM-RLB-17: Child Class Element without Virtual Destructor unlike its Parent Class Element
	\item ASCRM-RLB-18: Storable and Member Data Element Initialization with Hard-Coded Network Resource Configuration Data
	\item ASCRM-RLB-19: Synchronous Call Time-Out Absence
\end{itemize}




\begin{table}[h]
\textbf{Gestão de Configuração}
	\centering
	\caption{Questionário 1 - Processo - Parte 1}
	\begin{tabular}{p{3.5in}p{2in}}		
		\toprule
		\textbf{Questão}  & \textbf{Avaliação}\\ 
		\midrule
		1 - São identificados os itens de configuração, componentes e produtos de trabalho a serem 
colocados sob a gestão de configuração?
 & TI | IP | NI | NS \\
        \midrule
		2 - É estabelecido e mantido um sistema de gestão de configurações e gestão de alterações para 
controlar os produtos de trabalho?
 & TI | IP | NI | NS \\
		\midrule
		3 -  São criados ou lançados baselines para uso interno e para entrega ao cliente?
 & TI | IP | NI | NS \\
		\midrule
		4 - Os pedidos de alterações dos itens de configuração são acompanhados?
 & TI | IP | NI | NS \\
		\midrule
		5 – As alterações nos itens de configuração são controladas?
  & TI | IP | NI | NS \\
		\midrule
		6 – São estabelecidos e mantidos registos de gestão de configurações que descrevem os itens 
de configuração?
 & TI | IP | NI | NS \\
		\midrule
		7– Auditorias de configuração são realizadas para manter a integridade dos baselines?
 & TI | IP | NI | NS \\
		\bottomrule
	\end{tabular} 
	\label{tab:tabela1}
\end{table}

\begin{table}[h]
\textbf{Medição e Análise}
	\centering
	\caption{Questionário 1 - Processo - Parte 2}
	\begin{tabular}{p{3.5in}p{2in}}		
		\toprule
	
		\textbf{Questão}  & \textbf{Avaliação}\\ 
		\midrule
		8 – São estabelecidos objetivos de medição?
 & TI | IP | NI | NS \\
        \midrule
		9 – As medidas para satisfazer os objetivos de medição estão especificadas?
 & TI | IP | NI | NS \\
		\midrule
		10 – Existe uma especificação de como os dados de medição são obtidos e armazenados - 
(coleção de dados e respetivos procedimentos de armazenamento)?
 & TI | IP | NI | NS \\
		\midrule
        11 – Existe uma especificação de como os dados de medição são análise) e comunicados?
 & TI | IP | NI | NS \\
		\midrule
		12 – Os dados de medição especificados são obtidos?
  & TI | IP | NI | NS \\
		\midrule
		13 – São analisados e interpretados os dados resultantes da medição?
 & TI | IP | NI | NS \\
		\midrule
		14 – São geridos e armazenados os dados e resultados das medições, especificações de medição 
e resultados da análise efetuada?
 & TI | IP | NI | NS \\
        \midrule
        15 – Os resultados da medição e análise de atividades são comunicados a todas as partes
interessadas?
 & TI | IP | NI | NS \\
		\bottomrule
	\end{tabular} 
	\label{tab:tabela1}
\end{table}

\begin{table}[h]
\textbf{Monitorização e Controlo do Projeto}
	\centering
	\caption{Questionário 1 - Processo - Parte 3}
	\begin{tabular}{p{3.5in}p{2in}}		
		\toprule
		\textbf{Questão}  & \textbf{Avaliação}\\ 
		\midrule
		16 – São monitorizados os valores reais dos parâmetros do planeamento do projeto em relação 
ao plano de projeto? 
 & TI | IP | NI | NS \\
        \midrule
		17 – São monitorizados os compromissos em relação aos identificados no plano de projeto?
 & TI | IP | NI | NS \\
		\midrule
		18 – São Monitorizados os riscos em relação aos identificados no plano do projeto?
 & TI | IP | NI | NS \\
		\midrule
        19 – É monitorizada a gestão de dados do projeto em relação ao plano de projeto?
 & TI | IP | NI | NS \\
		\midrule
		20 – É monitorizado o envolvimento das partes interessadas em relação ao plano de projeto?
  & TI | IP | NI | NS \\
		\midrule
		21 – São revistos periodicamente o progresso, desempenho e as questões críticas do projeto?
 & TI | IP | NI | NS \\
		\midrule
		22 – São revistos em pontos-chave selecionados do projeto, as realizações do projeto e os 
resultados obtidos?
 & TI | IP | NI | NS \\
        \midrule
        23 – São identificadas e analisadas as questões críticas e determinadas as ações corretivas 
necessárias para tratar as mesmas?
 & TI | IP | NI | NS \\
		\midrule
		24 – São implementadas ações corretivas para tratar as questões críticas identificadas?
 & TI | IP | NI | NS \\
		\midrule
		25 – As ações corretivas são geridas até à sua conclusão?
 & TI | IP | NI | NS \\
		\bottomrule
	\end{tabular} 
	\label{tab:tabela1}
\end{table}

\begin{table}[h]
\textbf{Planeamento de Projeto}
	\centering
	\caption{Questionário 1 - Processo - Parte 4}
	\begin{tabular}{p{3.5in}p{2in}}		
		\toprule
		\textbf{Questão}  & \textbf{Avaliação}\\ 
		\midrule
		26 – É estabelecida uma estrutura analítica de projeto (WBS) de alto nível para estimar o âmbito 
do projeto?
 & TI | IP | NI | NS \\
        \midrule
		27 – São estabelecidas e mantidas estimativas para atributos de produtos de trabalho e de 
tarefas?
 & TI | IP | NI | NS \\
		\midrule
        28 – As fases do ciclo de vida do projeto são definidas?
 & TI | IP | NI | NS \\
		\midrule
        29 – As estimativas de esforço e de custo com base no raciocínio de estimativas são 
determinadas?
 & TI | IP | NI | NS \\
		\midrule
		30 – É estabelecido e mantido o orçamento e cronograma do projeto?
  & TI | IP | NI | NS \\
		\midrule
		31 – São identificados e analisados os riscos do projeto?
 & TI | IP | NI | NS \\
		\midrule
		32 – A gestão de dados do projeto é planeada?
 & TI | IP | NI | NS \\
        \midrule
        33 – Os recursos necessários para a execução do projeto são planeados?
 & TI | IP | NI | NS \\
        \midrule
         34 – Os conhecimentos necessários para a execução do projeto são planeados?
         & TI | IP | NI | NS \\
         \midrule
         35 – É planeado o envolvimento das partes interessadas identificadas?
         & TI | IP | NI | NS \\
         \midrule
         36 – É estabelecido e mantido o plano global do projeto?
         & TI | IP | NI | NS \\
         \midrule
         37 – Todos os planos que afetam o projeto para entender os compromissos do mesmo são 
revistos?
         & TI | IP | NI | NS \\
         \midrule
         38 – É ajustado o plano de projeto de acordo com os recursos estimados e disponíveis?
         & TI | IP | NI | NS \\
         \midrule
        39 – O compromisso das partes interessadas relevantes, responsáveis pela execução e apoio à 
execução do plano é obtido?
         & TI | IP | NI | NS \\
		\bottomrule
	\end{tabular} 
	\label{tab:tabela1}
\end{table}

\begin{table}[h]
\textbf{Garantia da Qualidade de Processo e Produto}
	\centering
	\caption{Questionário 1 - Processo - Parte 5}
	\begin{tabular}{p{3.5in}p{2in}}		
		\toprule
		\textbf{Questão}  & \textbf{Avaliação}\\ 
		\midrule
		40 – São avaliados objetivamente os processos selecionados em relação às descrições de 
processo, padrões e procedimentos aplicáveis?
 & TI | IP | NI | NS \\
        \midrule
		41 – São avaliados objetivamente os produtos de trabalho selecionados em relação às descrições 
de processo, padrões e procedimentos aplicáveis?
 & TI | IP | NI | NS \\
		\midrule
		42 – São comunicadas as questões críticas relativas à qualidade e asseguradas a resolução de 
não conformidades com a equipa e os gestores?
 & TI | IP | NI | NS \\
		\midrule
        43 – Os registos das atividades de garantia da qualidade são estabelecidos e mantidos?
 & TI | IP | NI | NS \\
		\bottomrule
	\end{tabular} 
	\label{tab:tabela1}
\end{table}

\begin{table}[h]
\textbf{Gestão de Requisitos}
	\centering
	\caption{Questionário 1 - Processo - Parte 6}
	\begin{tabular}{p{3.5in}p{2in}}		
		\toprule
		\textbf{Questão}  & \textbf{Avaliação}\\ 
		\midrule
		44 – É realizado um trabalho em conjunto, com quem definiu os requisitos de forma a obter um 
melhor entendimento do significado dos mesmos?
 & TI | IP | NI | NS \\
        \midrule
		45 – É obtido o compromisso com os participantes do projeto face aos requisitos?
 & TI | IP | NI | NS \\
		\midrule
		46 – As mudanças nos requisitos à medida que evoluem durante o projeto são geridas?
 & TI | IP | NI | NS \\
		\midrule
        47 – É mantida a rastreabilidade bidirecional dos requisitos e produtos de trabalho?
 & TI | IP | NI | NS \\
		\midrule
		48 – É garantido que os planos de projeto e produtos de trabalho permanecem alinhados com 
as exigências?
  & TI | IP | NI | NS \\
		\bottomrule
	\end{tabular} 
	\label{tab:tabela1}
\end{table}

\begin{table}[h]
\textbf{Gestão de Contrato com Fornecedores}
	\centering
	\caption{Questionário 1 - Processo - Parte 7}
	\begin{tabular}{p{3.5in}p{2in}}		
		\toprule
		\textbf{Questão}  & \textbf{Avaliação}\\ 
		\midrule
		49 – É determinado o tipo de aquisição para cada produto ou componente de produto a ser 
adquirido?
 & TI | IP | NI | NS \\
        \midrule
		50 – Os fornecedores são selecionados com base numa avaliação da sua capacidade em 
satisfazer os requisitos especificados e critérios estabelecidos?
 & TI | IP | NI | NS \\
		\midrule
		51 – São estabelecidos e mantidos contractos formais com os fornecedores?
 & TI | IP | NI | NS \\
		\midrule
        52 – São executadas atividades com o fornecedor conforme especificado no contrato com o 
mesmo?
 & TI | IP | NI | NS \\
		\midrule
		53 – Existe a verificação que o acordo com o fornecedor está satisfeito, antes de aceitar o 
produto adquirido?
  & TI | IP | NI | NS \\
		\midrule
		54 – É assegurada a transição dos produtos adquiridos no fornecedor?
 & TI | IP | NI | NS \\
		\bottomrule
	\end{tabular} 
	\label{tab:tabela1}
\end{table}

\begin{table}[h]
\textbf{Análise e Tomada de Decisões}
	\centering
	\caption{Questionário 1 - Processo - Parte 8}
	\begin{tabular}{p{3.5in}p{2in}}		
		\toprule
		\textbf{Questão}  & \textbf{Avaliação}\\ 
		\midrule
		55 – São estabelecidas e mantidas diretrizes para determinar quais as questões que são sujeitas 
a um processo formal?
 & TI | IP | NI | NS \\
        \midrule
		56 – São estabelecidos e mantidos critérios para avaliar as alternativas e para classificá-los de 
forma relativa?
 & TI | IP | NI | NS \\
		\midrule
		57 – As soluções alternativas para resolver problemas são identificadas?
 & TI | IP | NI | NS \\
		\midrule
        58 – São selecionados métodos de avaliação?
 & TI | IP | NI | NS \\
		\midrule
		59 – As soluções alternativas usando critérios e métodos estabelecidos são avaliadas?
  & TI | IP | NI | NS \\
		\midrule
		60 – São selecionadas soluções a partir das várias alternativas, com base nos critérios de 
avaliação?
 & TI | IP | NI | NS \\
		\bottomrule
	\end{tabular} 
	\label{tab:tabela1}
\end{table}

\begin{table}[h]
\textbf{Gestão Integrada do Projeto}
	\centering
	\caption{Questionário 1 - Processo - Parte 9}
	\begin{tabular}{p{3.5in}p{2in}}		
		\toprule
		\textbf{Questão}  & \textbf{Avaliação}\\ 
		\midrule
		61 – É estabelecido e mantido o processo definido para o projeto desde o início até ao fim do mesmo?
 & TI | IP | NI | NS \\
        \midrule
		62 – São utilizados os ativos do processo e o repositório de medições da organização para 
estimar e planear as atividades do projeto?
 & TI | IP | NI | NS \\
		\midrule
		63 – É estabelecido e mantido o ambiente de trabalho do projeto com base nos padrões de 
ambiente de trabalho da organização?
 & TI | IP | NI | NS \\
		\midrule
        64 – É integrado o plano do projeto e outros planos que afetam o projeto para descrever o 
processo definido do mesmo?
 & TI | IP | NI | NS \\
		\midrule
		65 – O projeto é gerido usando o plano de projeto, outros planos que afetam o projeto e o 
processo definido para o mesmo?
  & TI | IP | NI | NS \\
		\midrule
		66 – As equipas de projeto são estabelecidas e mantidas?
 & TI | IP | NI | NS \\
        \midrule
		67 – Existe a contribuição de experiências do processo relacionado com os processos ativos da 
organização?
 & TI | IP | NI | NS \\
        \midrule
		68 – É feita a gestão do envolvimento das partes interessadas?
 & TI | IP | NI | NS \\
        \midrule
		69  – Existe a colaboração com as partes interessadas na identificação, negociação e 
acompanhamento de dependências críticas?
 & TI | IP | NI | NS \\
        \midrule
		70 – São resolvidas questões críticas de coordenação com as partes interessadas?
 & TI | IP | NI | NS \\
		\bottomrule
	\end{tabular} 
	\label{tab:tabela1}
\end{table}

\begin{table}[h]
\textbf{Definição dos Processos da Organização}
	\centering
	\caption{Questionário 1 - Processo - Parte 10}
	\begin{tabular}{p{3.5in}p{2in}}		
		\toprule
		\textbf{Questão}  & \textbf{Avaliação}\\ 
		\midrule
		71 – São estabelecidos e mantidos processos-padrão da organização?
 & TI | IP | NI | NS \\
        \midrule
		72 – As descrições dos modelos do ciclo de vida aprovados para uso da organização são 
estabelecidas e mantidas?
 & TI | IP | NI | NS \\
		\midrule
		73 – Critérios e diretrizes para adaptação do conjunto de processos-padrão da organização são 
estabelecidos e mantidos?
 & TI | IP | NI | NS \\
		\midrule
        74 – O repositório de medições da organização é estabelecido e mantido?
 & TI | IP | NI | NS \\
		\midrule
		75 – A biblioteca de processos ativos da organização é estabelecida e mantida?
  & TI | IP | NI | NS \\
		\midrule
		76 – Os padrões de ambiente de trabalho são estabelecidos e mantidos?
 & TI | IP | NI | NS \\
        \midrule
        77 – Regras de organização e diretrizes para a estrutura, formação e funcionamento das equipas 
são estabelecidas e mantidas?
& TI | IP | NI | NS \\
		\bottomrule
	\end{tabular} 
	\label{tab:tabela1}
\end{table}

\begin{table}[h]
\textbf{Enfoque nos Processos da Organização}
	\centering
	\caption{Questionário 1 - Processo - Parte 11}
	\begin{tabular}{p{3.5in}p{2in}}		
		\toprule
		\textbf{Questão}  & \textbf{Avaliação}\\ 
		\midrule
		78 – A descrição das necessidades e objetivos dos processos da organização são estabelecidos e 
mantidos?
 & TI | IP | NI | NS \\
        \midrule
		79 – São avaliados periodicamente os processos da organização e conforme necessário, para 
conhecer os seus pontos fortes e fracos?
 & TI | IP | NI | NS \\
		\midrule
		80 – As melhorias para os processos da organização e ativos do processo são identificadas?
 & TI | IP | NI | NS \\
		\midrule
        81 – São estabelecidos e mantidos planos de ações do processo para promover melhorias nos 
processos e ativos de processo?
 & TI | IP | NI | NS \\
		\midrule
		82 – Os planos de ações de processo são implementados?82 – Os planos de ações de processo são implementados?
  & TI | IP | NI | NS \\
		\midrule
		83 – Os ativos de processos organizações são implantados em toda a organização?
 & TI | IP | NI | NS \\
        \midrule
		84 – O conjunto de processos-padrão são implantados nos projetos desde o início do mesmo, e 
a implementação de mudanças nesses processos ao longo do ciclo de vida de cada projeto
conforme apropriado são igualmente tornados comuns? 
 & TI | IP | NI | NS \\
 \midrule
		85 – A implementação do conjunto de processos-padrão da organização e o uso de ativos de
processos em todos os projetos é monitorizada? 
 & TI | IP | NI | NS \\\midrule
		86 – São incorporadas experiências em ativos de processos organizacionais?
 & TI | IP | NI | NS \\
		\bottomrule
	\end{tabular} 
	\label{tab:tabela1}
\end{table}

\begin{table}[h]
\textbf{Formação na Organização}
	\centering
	\caption{Questionário 1 - Processo - Parte 12}
	\begin{tabular}{p{3.5in}p{2in}}		
		\toprule
		\textbf{Questão}  & \textbf{Avaliação}\\ 
		\midrule
		87 – As necessidades de formação estratégicas da organização são estabelecidas e mantidas?
 & TI | IP | NI | NS \\
        \midrule
		88 – São determinadas quais as necessidades de formação são da responsabilidade da 
organização e quais devem ser atribuídas a projetos individuais ou grupos de trabalho?
 & TI | IP | NI | NS \\
		\midrule
		89 – Um plano tático de formação da organização é estabelecido e mantido?
 & TI | IP | NI | NS \\
		\midrule
        90 – A capacidade de formação para tratar das necessidades de formação na organização é 
estabelecida e mantida?
 & TI | IP | NI | NS \\
		\midrule
		91 – É fornecida formação de acordo com o plano tático de formação da organização?
  & TI | IP | NI | NS \\
		\midrule
		92 – Registos de formação da organização são estabelecidos e mantidos?
 & TI | IP | NI | NS \\
 \midrule
		93 – A eficácia do programa de formação da organização é avaliada?
 & TI | IP | NI | NS \\
		\bottomrule
	\end{tabular} 
	\label{tab:tabela1}
\end{table}

\begin{table}[h]
\textbf{Integração de Produto}
	\centering
	\caption{Questionário 1 - Processo - Parte 13}
	\begin{tabular}{p{3.5in}p{2in}}		
		\toprule
		\textbf{Questão}  & \textbf{Avaliação}\\ 
		\midrule
		94 – Uma estratégia de integração de produto é estabelecida e mantida?
 & TI | IP | NI | NS \\
        \midrule
		95 – O ambiente necessário para apoiar a integração dos componentes de produto é 
estabelecido e mantido?
 & TI | IP | NI | NS \\
		\midrule
		96 – Procedimentos e critérios para a integração dos componentes de produto são estabelecidos 
e mantidos?
 & TI | IP | NI | NS \\
		\midrule
        97 – São revistas as descrições de interface de forma a assegurar a cobertura e completude da 
mesma?
 & TI | IP | NI | NS \\
		\midrule
		98 – São geridas as definições de interfaces internos e externos, designs e mudanças de produtos 
e componentes de produto?
  & TI | IP | NI | NS \\
		\midrule
		99 – É confirmado se os componentes de produto estão prontos a serem integrados de acordo 
com a descrição, e as interfaces dos componentes de produto estão em conformidade com as
respetivas descrições? 
 & TI | IP | NI | NS \\
 \midrule
		100 – A integração e montagem dos componentes de produto estão de acordo com a estratégia
de integração do produto e procedimentos disponíveis? 
 & TI | IP | NI | NS \\
 \midrule
		101 – Existe a avaliação dos componentes de produto integrado e compatibilidade com as
interfaces? 
 & TI | IP | NI | NS \\
 \midrule
		102 – O produto integrado ou o componente do produto é transformado em package e é
entregue ao cliente? 
 & TI | IP | NI | NS \\
		\bottomrule
	\end{tabular} 
	\label{tab:tabela1}
\end{table}

\begin{table}[h]
\textbf{Desenvolvimento de Requisitos}
	\centering
	\caption{Questionário 1 - Processo - Parte 14}
	\begin{tabular}{p{3.5in}p{2in}}		
		\toprule
		\textbf{Questão}  & \textbf{Avaliação}\\ 
		\midrule
		103 – O levantamento das necessidades, expectativas, restrições e interfaces para todas as fases 
do ciclo de vida do produto é efetuado?
 & TI | IP | NI | NS \\
        \midrule
		104 – As necessidades, expectativas, restrições e interfaces são transformados em requisitos 
prioritários do cliente?
 & TI | IP | NI | NS \\
		\midrule
		105 – Os requisitos de produto e de componente de produto, com base nas necessidades do 
cliente são estabelecidos e mantidos?
 & TI | IP | NI | NS \\
		\midrule
        106 – Os requisitos de cada componente de produto são atribuídos?
 & TI | IP | NI | NS \\
		\midrule
		107 – Os requisitos de interface são identificados?
  & TI | IP | NI | NS \\
		\midrule
		108 – Conceitos operacionais e cenários associados são estabelecidos e mantidos?
 & TI | IP | NI | NS \\
 \midrule
		109 – Uma definição de funcionalidade e atributos de qualidade é estabelecida e mantida?
 & TI | IP | NI | NS \\
 \midrule
	110 – Os requisitos para assegurar que são necessários e suficientes são analisados?
 & TI | IP | NI | NS \\
 \midrule
		111 – Os requisitos para equilibrar as necessidades e as restrições das partes interessadas são 
analisados?
 & TI | IP | NI | NS \\
        \midrule
		112 – Os requisitos para assegurar que o produto final irá funcionar conforme o esperado no 
ambiente do utilizador final são validados?
 & TI | IP | NI | NS \\
		\bottomrule
	\end{tabular} 
	\label{tab:tabela1}
\end{table}

\begin{table}[h]
\textbf{Gestão de Risco}
	\centering
	\caption{Questionário 1 - Processo - Parte 15}
	\begin{tabular}{p{3.5in}p{2in}}		
		\toprule
		\textbf{Questão}  & \textbf{Avaliação}\\ 
		\midrule
		113 – Fontes e categorias de riscos são definidos?
 & TI | IP | NI | NS \\
        \midrule
		114 – Os parâmetros utilizados para analisar e categorizar os riscos e controlar o esforço de 
gestão de riscos são definidos?
 & TI | IP | NI | NS \\
		\midrule
		115 – Uma estratégia a ser utilizada para a gestão de riscos é estabelecida e mantida?
 & TI | IP | NI | NS \\
		\midrule
        116 – A identificação e documentação dos riscos é feita?
 & TI | IP | NI | NS \\
		\midrule
	117 – É avaliado e classificado cada risco identificado, utilizando as categorias e os parâmetros 
definidos para os riscos e determinada a sua prioridade relativa?
  & TI | IP | NI | NS \\
		\midrule
		118 – Um plano de atenuação de risco de acordo com a estratégia de gestão de riscos é 
desenvolvido?
 & TI | IP | NI | NS \\
 \midrule
		119 – É Monitorizado periodicamente o status de cada risco, e implementado o plano de 
atenuação de risco quando necessário?
 & TI | IP | NI | NS \\
		\bottomrule
	\end{tabular} 
	\label{tab:tabela1}
\end{table}

\begin{table}[h]
\textbf{Solução Técnica}
\centering
	\caption{Questionário 1 - Processo - Parte 16}
	\begin{tabular}{p{3.5in}p{2in}}		
		\toprule
		\textbf{Questão}  & \textbf{Avaliação}\\ 
		\midrule
		120 – Soluções alternativas e critérios de seleção são desenvolvidos?
 & TI | IP | NI | NS \\
        \midrule
		121 – São selecionadas as soluções associadas a componentes de produto baseadas em critérios 
de seleção?
 & TI | IP | NI | NS \\
		\midrule
		122 – Um design para o produto ou componente de produto é desenvolvido?
 & TI | IP | NI | NS \\
		\midrule
        123 – Um pacote de dados técnicos é estabelecido e mantido?
 & TI | IP | NI | NS \\
		\midrule
		124 – Projetar as interfaces dos componentes de produto usando critérios estabelecidos é 
realizado? 
  & TI | IP | NI | NS \\
		\midrule
		125 – É avaliado se os componentes de produto devem ser desenvolvidos, comprados ou 
reutilizados com base em critérios estabelecidos? 
 & TI | IP | NI | NS \\
 \midrule
		126 – O design dos componentes de produto é implementado?
 & TI | IP | NI | NS \\
        \midrule
        127 – A elaboração e atualização da documentação para o utilizador final são realizadas?
 & TI | IP | NI | NS \\
		\bottomrule
	\end{tabular} 
	\label{tab:tabela1}
\end{table}

\begin{table}[h]
\textbf{Validação}
	\centering
	\caption{Questionário 1 - Processo - Parte 17}
	\begin{tabular}{p{3.5in}p{2in}}		
		\toprule
		\textbf{Questão}  & \textbf{Avaliação}\\ 
		\midrule
		128 – São selecionados os produtos ou componentes de produto a serem validados e os 
métodos de validação a serem utilizados?
 & TI | IP | NI | NS \\
        \midrule
		129 – O ambiente necessário para suportar a validação é estabelecido e mantido?
 & TI | IP | NI | NS \\
		\midrule
		130 – Os procedimentos e critérios de validação são estabelecidos e mantidos?
 & TI | IP | NI | NS \\
		\midrule
        131 – É realizada a validação dos produtos e componentes de produto selecionados?
 & TI | IP | NI | NS \\
		\midrule
		132 – Os resultados das atividades de validação são analisados?
  & TI | IP | NI | NS \\
		\bottomrule
	\end{tabular} 
	\label{tab:tabela1}
\end{table}

\begin{table}[h]
\textbf{Verificação }
	\centering
	\caption{Questionário 1 - Processo - Parte 18}
	\begin{tabular}{p{3.5in}p{2in}}		
		\toprule
		\textbf{Questão}  & \textbf{Avaliação}\\ 
		\midrule
		133 – Os produtos de trabalho a serem verificados e os métodos de verificação a serem 
utilizados são selecionados?
 & TI | IP | NI | NS \\
        \midrule
		134 – O ambiente necessário para suportar a verificação é estabelecido e mantido?
 & TI | IP | NI | NS \\
		\midrule
		135 – Os procedimentos e critérios de verificação para os produtos de trabalho selecionados são 
estabelecidos e mantidos?
 & TI | IP | NI | NS \\
		\midrule
        136 – Existe preparação para a revisão por pares dos produtos de trabalho selecionados?
 & TI | IP | NI | NS \\
		\midrule
		137 – É realizada a revisão por pares nos produtos de trabalho selecionados e identificados 
problemas resultantes da revisão efetuada?
  & TI | IP | NI | NS \\
		\midrule
		138 – Os dados sobre a preparação, realização e resultados obtidos destas revisões são 
analisados?
 & TI | IP | NI | NS \\
 \midrule
		139 – A verificação nos produtos de trabalho selecionados é realizada?
 & TI | IP | NI | NS \\
  \midrule
		140 – Os resultados de todas as atividades de verificação são analisados? 
 & TI | IP | NI | NS \\
		\bottomrule
	\end{tabular} 
	\label{tab:tabela1}
\end{table}

%\begin{figure}[h]
%	\centering
%	\includegraphics[width=0.4\linewidth]{}
%	\caption{Exemplo de imagem: \small{\textit{Fonte: 	\cite{CMMI2010}}}}
%	\label{fig:figura1}
%\end{figure}


\end{appendix}
\end{document}